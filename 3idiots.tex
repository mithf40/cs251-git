\documentclass{article}


\usepackage[scale=0.75]{geometry}


\title{3 Idiots}
\begin{document}
\begin{center} \LARGE 3 IDIOTS
\end{center}

\Large{
\begin{center}\textbf{3 Idiots} is movie showing lesson on study hard, and attain good grades. 3 Idiots is a 2009 Indian coming of age comedy-drama film co-written, edited, and directed by Rajkumar Hirani and produced by Vidhu Vinod Chopra. Abhijat Joshi wrote the screenplay. It was inspired by the novel Five Point Someone by Chetan Bhagat.[4] The film stars Aamir Khan, Kareena Kapoor, R. Madhavan, Sharman Joshi, Omi Vaidya, Parikshit Sahni and Boman Irani.

Upon release, the film was the highest-grossing film in its opening weekend in India and had the highest opening day collections for an Indian film up until that point. It also held the record for the highest net collections in the first week for a Bollywood film. It also became one of the few Indian films to become successful at the time in East Asian markets such as China,[5] eventually bringing its overseas total to more than US65 million at the time—the highest-grossing Bollywood film of all time in overseas markets, before being overtaken by Chennai Express in 2013. But according to a 2015 survey and viewers interview, it is said that if 3 Idiots would have released in 2015, it would have earned around ₹5 billion. Over the years, it has attained the cult status.

The film is distinctive for featuring real inventions by little-known people in India's backyards. The brains behind the innovations were Remya Jose, a student from Kerala, who created the pedal operated washing-machine;[6][7][8] Mohammad Idris, a barber from Hasanpur Kalan in Meerut district in Uttar Pradesh, who invented a bicycle-powered horse clipper;[9] and Jahangir Painter, a painter from Maharashtra, who made the scooter-powered flour mill.[10] The subtitled version of the film grew popular in South Asia, especially China, Malaysia, Singapore and the Philippines.[11] \end{center}}
\section{Plot}
Farhan Qureshi (Madhavan) and Raju Rastogi (Joshi) are students at the prestigious Imperial College of Engineering (ICE) who also share a room at the institution's dormitories, with Man "Millimeter" Mohan (Kumar) serving as the dormitory household serviceman. Farhan's passion is wildlife photography, but he pursues an engineering degree to appease his father. Raju chooses engineering with hopes of improving his family's financial situation, but his lack of self-confidence results in poor grades. Their third roommate and friend, Ranchodas "Rancho" Chanchad (Khan), has an immense interest in engineering, and invents in his spare time. After giving unorthodox answers in class, Rancho faces scrutiny from the college's director, Dr. Viru Sahastrabuddhi (Irani), known as "Virus", whose traditional philosophies on education and learning contrast sharply with Rancho's atypical ideas of teaching. Virus is shown to be extremely strict — a trait that indirectly caused his own son's suicide after putting him under excessive pressure.

Rancho eventually falls in love with Pia (Kapoor), Virus' younger daughter, who is a medical student in residency at the city's hospital. Rancho always scores the highest marks on exams out of his class, much to the dismay of both Virus and Chatur Ramalingam (Vaidya), an arrogant Ugandan-born Tamil student who believes in rote learning, and sneers the trio. Chatur, however, is humiliated when he delivers a Hindi speech modified by Rancho in front of the ICE community, the Education Minister and Virus during a Teachers' Day event.

One night, Rancho, Farhan, and Raju drunkenly break into the Sahastrabuddhe household to allow Rancho to profess his love to Pia. After discovering their antics, Virus threatens to expel Raju unless he writes a letter blaming Rancho for the break-in. Unwilling to betray Rancho or disappoint his family, Raju unsuccessfully attempts suicide, and ends up in a coma. After intensive care and support from his friends, Raju recovers just before he successfully interviews for a corporate job. Before this Rancho and Pia posted Farhan's letter to his favorite photographer. Farhan gets a positive response but fears his father. Rancho then convinces him to go confront his father and his father reluctantly agrees to it.

Unfortunately, Virus gets frustrated at Rancho's influence on Raju and Farhan after knowing all these, and he conspires to deliberately modify the final exams and make it so difficult so that none of them can graduate. However, Pia gives Virus' office duplicate keys to Rancho to enable him to get the exam papers. But Virus finds the trio and expels them. The trio, however, earn a reprieve when Virus' pregnant elder daughter Mona (Singh) goes into labour at the same time a heavy storm cuts off all power and traffic. Despite this, Rancho uses his engineering knowledge to deliver the baby in the college common room. A grateful Virus finally acknowledges Rancho as an extraordinary student, and allows the three to graduate. Rancho then unexpectedly disappears shortly after the ceremony. None of them have heard from Rancho since graduation.

Ten years later, Farhan is a successful wildlife photographer, Raju is doing corporate job and settled in a comfortable lifestyle with his wife and Chatur Ramalingam is a vice president of a corporation in the United States. In present-day Farhan is boarding for a flight. Suddenly he gets a call from Chatur telling him that he found Rancho. Farhan causes an emergency landing by faking a heart attack. On the way to the campus, he takes Raju and reaches the campus only to find Chatur but not Rancho. Chatur reveals that Rancho is in Shimla.

Arriving at Shimla, they head to the Chanchadd household where they meet a man with the same name (Jaffrey). From him, they learn that the Rancho they knew was actually "Chhote", an orphaned servant to the household who had passion for learning. When the real Ranchodas went to London for four years, his father allowed the servant to fill his son's place - including using his son's full name - and take credits for the degree. Ranchodas gives the address of "Chhote" in Ladakh, where he is a school teacher. On the way, the three rescue Pia from her arranged wedding with Suhas Tandon (Olivier Sanjay Lafont) at Manali.

Upon arrival in Ladakh, the four find the village school where they see young students' inventions resembling Rancho's own college projects. After meeting with Man Mohan, who is now known as "Centimeter" (Wagh), Raju, Farhan, and Pia then happily reunite with Rancho on a sandbar, where Rancho and Pia kiss. Assuming Rancho to just be a school teacher, Chatur forces him to concede that he is less successful than Chatur. Shortly after, Rancho reveals that he became a scientist while also teaching young children when he's not researching, and that his real name is Phunsukh Wangdu. Much to Chatur's horror, this also happens to be the name of an important inventor and business magnate that Chatur had spent a year trying to find and sign a business deal with. He concedes defeat to them.


\section{Production}
Principle photography began on 28 juy 2008. Hirani and his team left in late august for shoot with the principle cast. The film was shot in Delhi, Bangalore, Mumbai, Ladakh and Shimla. 



\section{Cast}
\begin{itemize}
    \item{Aamir Khan as Ranchoddas "Rancho" Shamaldas Chanchad/Chhote/Phunsukh Wangdu – one of the title group of three friends in the engineering college. He went missing after graduation and after 10 years his two friends traveled across India looking for him, while telling stories of their time in engineering college together. Rancho, as a student, was intelligent and had a brilliant personal philosophy. He rallied against unjust systems of teaching. At the end of the film, he is shown to be a famous scientist, entrepreneur and business magnate who also teaches young children when he takes a break from researching}
    \item{Shoaib Ahmed as young Chhote}
    \item{R. Madhavan as Farhan Qureshi – the film's narrator and a friend of Rancho and Raju. His father wanted him to be an engineer despite his lack of interest in the career. Instead, he becomes an accomplished wildlife photographer}
    \item{Sharman Joshi as Raju Rastogi. He comes from an impoverished family with a mother who is a retired school teacher and a paralyzed father who had been a postman. In the flashback story, his family is poor so they can't afford the car that would be demanded as a dowry for his sister. In the present story, he is a settled married man in Delhi who has freed his family from poverty by becoming a wealthy executive}
    \item{Kareena Kapoor as Pia Sahastrabuddhe – Viru "Virus" Sahastrabuddhe's younger daughter, an intelligent and capable doctor. Rancho is her love interest, and she breaks off her engagement to another to be with Rancho}
    \item{Boman Irani as Dr Viru Sahastrabuddhe – better known as "Virus", he is the strict college director. He is also Pia's father, and the film's "antagonist". By the end of the movie, he has changed his doctrinal method of teaching}
    \item{Omi Vaidya as Chatur Ramalingam – better known as "Silencer". A Ugandan-born Tamil who is the opposite of Rancho, Farhan, and Raju. Chatur is depicted as having a mere inability to speak Hindi due to two factors; being born in Uganda and having completed basic education in Pondicherry.[clarification needed] He believes in mindless memorisation. and sneers at Rancho's distinctive ideas, as does Virus. In the present story, he is vice-president of an American company who only discovers his success being overshadowed by Rancho in the end of the film}
    \item{Baradwaj Rangan of the New Indian Express wrote that Chatur being a Tamil from Uganda makes him "twice removed from the North Indians around him — a stranger to the nation as well as the national language}
    \item{Rahul Kumar as Manmohan – Better known as "Millimeter" or "MM", a young man who earns a small living in the college such as helping students by ironing their clothes, finishing assignments, and getting groceries; Rancho persuades him to buy a school uniform and go to any school to gain knowledge}
    item{Dushyant Wagh as Centimeter/Elder Manmohan – the present-day Millimeter who becomes Centimeter, who works as Rancho's/Phunsukh Wangdu's assistant in Ladakh}
    \item{Mona Singh as Mona Sahastrabuddhe – Pia's elder sister and Virus's first daughter}
    \item{Parikshit Sahni as Mr Qureshi – Farhan's father, a strict but loving parent who just wants his son to be happy}
    \item{Amardeep Jha as Mrs Rastogi – Raju's mother, a retired schoolteacher and dedicated mother}
    \item{Javed Jaffrey as the real Ranchoddas Shamaldas Chanchad – a person Raju, Farhan, and Chatur meet during the Shyamaldas Chanchad funeral service. His character is shown to be a corrupt person right from childhood, taking benefits from 'Chhote' in his homework and his exams. His father sends him to London and he sends 'Chhote' to ICE to gain an engineering degree in his name. He does appreciate what Chhote did for him, giving Raju and Farhan information on where to find him}
    \item{Arun Bali as Shamaldas Chanchad – father of Ranchoddas Shamaldas Chanchad}
    \item{Ali Fazal as Joy Lobo – a student with a passion for machines. After Virus tells him that he will not graduate, he commits suicide}
    \item{Akhil Mishra as Librarian Dubey}
    \item{Rohitash Gaud as Ranchoddas' servant}
    \item{Achyut Potdar as Machine Class Professor}
    \item{Madhav Vaze as Joy Lobo's father.}
    \item{Olivier Sanjay Lafont as Suhas Tandon – a materialistic man who is Pia's ex-fiance.}
    \item{Jayant Kripalani as Interviewer – the company head who conducts Raju's job interview}
\end{itemize}

\end{document}

